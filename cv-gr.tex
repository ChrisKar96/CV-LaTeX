%!TeX spellcheck = el_GR-en_US
%!TEX TS-program = xelatex

\documentclass{mycv}

\RequirePackage[greek]{datetime2} 		 % show greek date correctly using \today command
\renewcommand{\today}{\ifcase\month\or
	Ιανουάριος\or Φεβρουάριος\or Μάρτιος\or% 
	Απρίλιος\or Μάιος\or Ιούνιος\or Ιούλιος%
	\or Αύγουστος\or Σεπτέμβριος\or% 
	Οκτώβριος\or Νοέμβριος\or% 
	Δεκέμβριος\fi\ \number \year}			 % nominative instead of genitive 
\RequirePackage{amsmath}

\hypersetup{%
	pdftitle={Χρήστος Καραμολέγκος - Βιογραφικό Σημείωμα},
	pdfauthor={Χρήστος Καραμολέγκος},
	pdfsubject={Βιογραφικό Σημείωμα},
	pdfkeywords={Χρήστος Καραμολέγκος,Βιογραφικό Σημείωμα,CV},
	pdflang={el-GR}
}

\begin{document}
	\pagestyle{empty}
	\begin{minipage}{.7\textwidth}
		\begin{flushleft}
			\name{Χρήστος Καραμολέγκος}{Μηχανικός Λογισμικού}{Systems Administrator}
			\contact{(+30) 694 967 4952}{me@christoskaramo.tk}{ChrisKaramo}{www.christoskaramo.tk}{linkedin.com/in/chriskaramo}{ChrisKar96}
			\centering
			{\bf Ημερομηνία Γέννησης}: 24 Μαΐου 1996 {\Large\textperiodcentered} {\bf Τοποθεσία}: Αθήνα, Ελλάδα
		\end{flushleft}
	\end{minipage}
	\begin{minipage}{.3\textwidth}
		\begin{flushright}
			\includegraphics[scale=0.05]{assets/christos.png}
		\end{flushright}
	\end{minipage}
	%
	\vspace*{-0.5cm}
	%
	\section{Εισαγωγικό Σημείωμα}
	\textnormal Το όνομά μου είναι Χρήστος Καραμολέγκος. Είμαι τελειόφοιτος του τμήματος Ηλεκτρολόγων Μηχανικών και Μηχανικών Υπολογιστών του Πανεπιστημίου Δυτικής Μακεδονίας. Γεννήθηκα το 1996 στην Αθήνα και ζω στην Ελλάδα από τότε. Μιλάω εγγενώς ελληνικά και αγγλικά ως τη δεύτερη γλώσσα μου. Από μικρή ηλικία ανακάλυψα την αγάπη μου για τον προγραμματισμό και τους υπολογιστές γενικότερα. Πάθος μου είναι η επίλυση προβλημάτων και ο σχεδιασμός λύσεων. Προσπαθώ να μαθαίνω συνεχώς περισσότερα και να αποκτώ δεξιότητες, χρησιμοποιώντας τα πιο ενημερωμένα εργαλεία. Οι τομείς εξειδίκευσης που με ενδιαφέρουν περισσότερο είναι στον χώρο των Systems Administration και DevOps.

	\section{Εμπειρία}

	\begin{EntryDatedLogo}{ΕΕΛΛΑΚ - ΟΡΓΑΝΙΣΜΟΣ ΑΝΟΙΧΤΩΝ ΤΕΧΝΟΛΟΓΙΩΝ}{https://eellak.gr}{Ιούλιος 2019 -- Τώρα}{Systems Administrator}{-0.3cm}{assets/eellak.png}{0.75}
		\begin{Itemize}
			\item Υπεύθυνος για τη συντήρηση, τη διαμόρφωση και την αξιόπιστη λειτουργία της υποδομής του οργανισμού.
			\item Τα καθήκοντα συμπεριλαμβάνουν διαχείριση βάσεων δεδομένων, δικτύου, ασφάλειας, ιστού και διακομιστών.
			\item Η εγκατάσταση αποτελείται από 100+ on-premise or cloud VMs, κυρίως με \href{https://www.debian.org/}{\textit{Debian}} OS, διαχειριζόμενα με χρήση του \href{https://debops.org}{\textit{DebOps}} ansible project.
			\item Εγκατάσταση, παραμετροποίηση και διαχείριση self-hosted υπηρεσιών όπως \href{https://wordpress.com}{\textit{WordPress}}, \href{https://moodle.org}{\textit{Moodle}}, \href{https://bigbluebutton.org}{\textit{BigBlueButton}}, \href{https://nextcloud.com}{\textit{Nextcloud}}, \href{https://matrix.org}{\textit{Matrix}} / \href{https://element.io}{\textit{Element}} / \href{https://jitsi.org}{\textit{Jitsi}}.
		\end{Itemize}
	\end{EntryDatedLogo}

	\begin{EntryDated}{Ελεύθερος Επαγγελματίας}{https://www.christoskaramo.tk}{2018 -- Τώρα}{Προγραμματιστής}{-1cm}
	\begin{Itemize}
		\item Εργασίες ανάπτυξης λογισμικού και ιστοσελίδων, για ιδιώτες και επιχειρήσεις.
	\end{Itemize}
	\end{EntryDated}

	\vspace*{-1.2cm}

	\begin{EntryDated}{}{}{}{Οικοδιδάσκαλος}{-1cm}
		\begin{Itemize}
			\item Παράδοση ιδιαίτερων μαθημάτων σε πανεπιστημιακά τμήματα C/C++, Java, Προγραμματισμός Διαδικτύου, Βάσεις Δεδομένων, Τεχνητή Νοημοσύνη, Ανάπτυξη και Σχεδίαση Αλγορίθμων, Τεχνολογία Λογισμικού
		\end{Itemize}
	\end{EntryDated}

	\section{Δεξιότητες}
	\begin{tabular}{m{4.5cm} m{13cm}}
		\textbf{Διάγνωση / Επιδιόρθωση}     & Διάγνωση και επίλυση προβλημάτων υλικού και λογισμικού. \\
		\textbf{Διαχείριση Συστημάτων}		& Unix, NGINX, PKI, SSH, Email, LDAP, OpenVPN, VMware/Virtualbox, Ansible \\
		\textbf{DevOps}	                    & Git, Github, GitLab, TravisCI, Docker \\
		\textbf{Προγραμματισμός} 	 	  	& C/C++, Java, Python, Bash Scripting, \LaTeX \\
		\textbf{Προγρ. Διαδικτύου}	  		& HTML/CSS, Bootstrap, Javascript, JQuery, Ajax, PHP, SQL, Wordpress \\
		\textbf{Λειτουργικά Συστήματα}   	& Windows, Linux, Android \\
		\textbf{Διάφορα}        		 	& Σουίτες Office, Αλγοριθμική Σχεδίαση, Οργάνωση Project, Άδεια Οδήγησης Κατηγορίας Β \\
		\textbf{Γλώσσες} 			   		& Ελληνικά (μητρική γλώσσα), Αγγλικά 
	\end{tabular}

	\section{Εκπαίδευση}

	\begin{EntryDatedLogo}{Πανεπιστήμιο Δυτικής Μακεδονίας}{https://ece.uowm.gr}{2014 -- Τώρα}{Τμήμα Ηλεκτρολόγων Μηχανικών και Μηχανικών Υπολογιστών}{-0.3cm}{assets/uowm.pdf}{0.6}
	\begin{Itemize}
		\item Εν μέσω σπουδών για την απόκτηση του διπλώματός μου.
	\end{Itemize}
	\end{EntryDatedLogo}

	\section{Πιστοποιήσεις}
    \begin{EntryDatedLogo}{GitLab Certified Associate}{https://about.gitlab.com/services/education/gitlab-certified-associate/}{Ιούλιος 2021 -- Τώρα}{GitLab Associate}{-0.3cm}{assets/GitLab-Certified-Associate-2021-07-09.png}{0.90}
		\begin{Itemize}
			\item Άτομα που κερδίζουν πιστοποίηση GitLab Certified Associate, μπορούν να εξηγήσουν τι είναι το GitLab και γιατί το χρησιμοποιούν οι ομάδες, να εκτελούν βασικές εντολές Git για branching, merging και remote work και να εφαρμόζουν θεμελιώδεις έννοιες και δεξιότητες χρησιμοποιώντας το GitLab στον κύκλο ζωής του DevOps. 
			\item Επικύρωση στο \href{https://gitlab.badgr.com/public/assertions/Hw6j8Th9SyKNj8ehsQkqAw}{\textit{gitlab.badgr.com}}.
		\end{Itemize}
	\end{EntryDatedLogo}

	\vspace{0.75cm}

	\begin{EntryDatedLogo}{WSO2 Certified Identity Server
			Developer - V5}{https://wso2.com/training/certification/certified-identity-server-developer}{Ιούνιος 2020 -- Τώρα}{WSO2 Identity Server Developer}{-0.3cm}{assets/wso2is-cert.pdf}{0.90}
		\begin{Itemize}
			\item Αυτή η πιστοποίηση έχει σχεδιαστεί για προγραμματιστές και αρχιτέκτονες εφαρμογών που έχουν θεμελιώδη γνώση των εννοιών IAM και πρακτική εμπειρία με τον WSO2 Ιdentity Server. 
			\item Επικύρωση στο \href{https://certification.wso2.com}{\textit{certification.wso2.com}}. ID Πιστοποίησης: \textit{4TKN8V}
		\end{Itemize}
	\end{EntryDatedLogo}

	\section{Εθελοντική Εργασία}
	\begin{EntryDatedLogo}{Κέντρο Κοινωνικής Πρόνοιας Κεντρικής Μακεδονίας}{http://www.kkp-km.gr/}{2018 -- 2019}{Ανάπτυξη Λογισμικού}{-0.3cm}{assets/kkpkm.pdf}{0.75}
		\begin{Itemize}
			\item Βασικός developer και σχεδιαστής ενός web application που καταγράφει και διαχειρίζεται τις θεραπείες των περιθαλπόμενων ενός ιδρύματος.
			\item Περισσότερες πληροφορείες στο site του  \href{https://diavgeia.gov.gr/decision/view/\%CE\%A8\%CE\%A6\%CE\%A1\%CE\%93\%CE\%9F\%CE\%9E\%CE\%A7\%CE\%A3-\%CE\%A0\%CE\%93\%CE\%A6}{\textit{diavgeia}}.
		\end{Itemize}
	\end{EntryDatedLogo}

\end{document}