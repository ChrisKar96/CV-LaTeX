%!TeX spellcheck = el_GR-en_US
%!TEX TS-program = xelatex

\documentclass{mycv}

\RequirePackage[greek]{datetime2} 		 % show greek date correctly using \today command
\renewcommand{\today}{\ifcase\month\or
	Ιανουάριος\or Φεβρουάριος\or Μάρτιος\or% 
	Απρίλιος\or Μάιος\or Ιούνιος\or Ιούλιος%
	\or Αύγουστος\or Σεπτέμβριος\or% 
	Οκτώβριος\or Νοέμβριος\or% 
	Δεκέμβριος\fi\ \number \year}			 % nominative instead of genitive 
\RequirePackage{amsmath}

\hypersetup{
	pdftitle={Χρήστος Καραμολέγκος - Βιογραφικό Σημείωμα},
	pdfauthor={Χρήστος Καραμολέγκος},
	pdfsubject={Βιογραφικό Σημείωμα},
	pdfkeywords={Χρήστος Καραμολέγκος,Χρήστος,Καραμολέγκος,Βιογραφικό Σημείωμα,CV},
	pdflang={el-GR}
}

\begin{document}
	\pagestyle{empty}
	\begin{center}
		\name{Χρήστος Καραμολέγκος}{DevOps Engineer}{Systems Administrator}
		\contact{(+30)694-967-4952}{me@ckaramolegkos.gr}{ChrisKaramo}{www.ckaramolegkos.gr}{/in/chriskaramo}{ChrisKar96}
		\textbf{Ημερομηνία Γέννησης}: 24 Μαΐου 1996 {\Large\textperiodcentered} \textbf{Τοποθεσία}: Αθήνα, Ελλάδα
	\end{center}
	
	\section{Δεξιότητες}
	\begin{tabular}{m{4.5cm} m{12.5cm}}
	\textbf{Διαχείριση Συστημάτων} & Apache2, Nginx, PKI, SSH, Email, LDAP, OpenVPN, Virtualization, High Availability \\
	\textbf{DevOps}	               & Ansible, Git, Github, Github Actions, GitLab, TravisCI, Docker, Podman, Kubernetes \\
	\textbf{Monitoring}            & Zabbix, Icinga 2, Nagios, Prometheus, Grafana, Monit, Munin, NetData, Uptime Kuma \\
	\textbf{Προγραμματισμός} 	   & Bash Scripting, Python, Jinja2, YAML, TOML, Go, \LaTeX \\
	\textbf{Προγρ. Διαδικτύου}	   & HTML/CSS, Bootstrap, Javascript, JQuery, Ajax, PHP, SQL, WordPress, Redis \\
	\textbf{Γλώσσες} 			   & Ελληνικά (μητρική γλώσσα), Αγγλικά 
	\end{tabular}

	\section{Εμπειρία}

	\begin{EntryDatedLogo}{Power Factors}{https://powerfactors.com}{Ιανουάριος 2024 -- Τώρα}{Senior ITSM Engineer}{assets/powerfactors.png}{0.65}
		\begin{Itemize}
			\item Αυτοματοποίησα και πραγματοποίησα εγκατάσταση και συντήρηση σε ένα στόλο περισσότερων από 7500 διακομιστών και συσκευών σε όλο τον κόσμο.
			\item Συντήρησα εκατοντάδες ζωντανές υπηρεσίες παρακολουθώντας δείκτες επιδόσεων όπως διαθεσιμότητα, ρυθμό μετάδοσης και καθυστέρηση.
			\item Ανέπτυξα τα συστήματα υποστηρίζοντας αλλαγές που ενισχύουν την αξιοπιστία, την απόδοση και τη δυνατότητα εξυπηρέτησης.
			\item Συνεργάστηκα με την ομάδα ανάπτυξης για την τελειοποίηση των διαδικασιών CI/CD και την υλοποίηση δεκάδων υπηρεσιών, συμπεριλαμβανομένων πακέτων και containers, τόσο στο cloud όσο και σε on-premises Kubernetes.
		\end{Itemize}
		\vspace{-0.3cm}
		\textit{Relevant Skills}: \textbf{Linux}, \textbf{Bash}, \textbf{Ansible}, \textbf{VPN}, \textbf{Monitoring}, \textbf{Kafka}, \textbf{MQTT}, \textbf{Cassandra}, \textbf{Docker}, \textbf{Kubernetes}
	\end{EntryDatedLogo}
	
	\vspace{0.5cm}

	\begin{EntryDatedLogo}{Ελληνικός Στρατός}{http://army.gr}{Μαρτιος 2023 -- Δεκέμβριος 2023}{Υποχρεωτική Στρατιωτική Θητεία}{assets/hellenicarmy.png}{0.75}
		\begin{Itemize}
			\item Εκσυγχρόνισα μια κρυπτογραφικά ασφαλή πλατφόρμα κινητών επικοινωνιών μεταξύ περισσότερων από 650 απομακρυσμένων συσκευών σε όλη την Ελλάδα.
			\item Τυποποίησα και τεκμηρίωσα δεκάδες διαδικασίες, επιτρέποντας την εκπαίδευση τόσο των χρηστών όσο και των διαχειριστών στη λειτουργία της πλατφόρμας.
		\end{Itemize}
		\vspace{-0.3cm}
		\textit{Σχετικές Δεξιότητες}: \textbf{Linux}, \textbf{VPN}, \textbf{SIP}, \textbf{Monitoring}, \textbf{High Availability}, \textbf{Python / Django}, \textbf{Docker}, \textbf{Android}
	\end{EntryDatedLogo}

	\vspace{0.5cm}

	\begin{EntryDatedLogo}{Χίλων Πληροφορική}{https://web.archive.org/web/20220522083934/http://www.hilonsys.com/}{Μάιος 2022 -- Μάρτιος 2023}{DevOps Engineer}{assets/hilonsys.png}{0.75}
		\begin{Itemize}
			\item Υποστήριξα την εισαγωγή εργαλείων και διαδικασιών που σέβονται τις βέλτιστες πρακτικές DevOps, όπως οι \href{https://12factor.net/}{\textit{twelve factors}}.
			\item Αναθεώρησα την υποδομή του οργανισμού, υλοποιώντας δεκάδες μόνο μέσω VPN προσβάσιμες, dockerized υπηρεσίες, όπως διακομιστής OpenVPN, NTP, εσωτερικός/εξωτερικός διακομιστής DNS, GitLab, Ansible Controller, MTA (Postfix) και άλλες.
			\item Ξεκίνησα δύο αυτοδιαχειριζόμενες συστάδες Kubernetes χρησιμοποιώντας Ansible και το \href{https://kubespray.io/}{\textit{kubespray}}, που χρησιμοποιούνται τόσο για σκοπούς ανάπτυξης όσο και παραγωγής.
		\end{Itemize}
		\vspace{-0.3cm}
		\textit{Σχετικές Δεξιότητες}: \textbf{Linux}, \textbf{Ansible}, \textbf{VPN}, \textbf{Bash}, \textbf{PKI}, \textbf{LDAP}, \textbf{Docker}, \textbf{Kubernetes}
	\end{EntryDatedLogo}

	\vspace{0.5cm}

	\begin{EntryDatedLogo}{ΕΕΛΛΑΚ - ΟΡΓΑΝΙΣΜΟΣ ΑΝΟΙΧΤΩΝ ΤΕΧΝΟΛΟΓΙΩΝ}{https://eellak.gr}{Ιούλιος 2019 -- Μάρτιος 2023}{Systems Administrator}{assets/eellak.png}{0.75}
		\begin{Itemize}
			\item Διαχειρίστηκα την υποδομή του οργανισμού με πάνω από 130 VMs, σε physical on-premise ή cloud hypervisors, κυρίως με \href{https://www.debian.org}{\textit{Debian}} OS, ελεγχόμενο με χρήση του \href{https://debops.org}{\textit{DebOps}} Ansible project.
			\item Εγκατέστησα, παραμετροποίησα και διαχειρίστηκα εκατοντάδες self-hosted υπηρεσίες όπως \href{https://wordpress.com}{\textit{WordPress}}, \href{https://moodle.org}{\textit{Moodle}}, \href{https://bigbluebutton.org}{\textit{BigBlueButton}}, \href{https://about.gitlab.com/install/}{\textit{GitLab}}, \href{https://nextcloud.com}{\textit{Nextcloud}}, \href{https://matrix.org}{\textit{Matrix}} / \href{https://element.io}{\textit{Element}} / \href{https://jitsi.org}{\textit{Jitsi}}, \href{https://www.redmine.org/}{\textit{Redmine}}, \href{https://vaultwarden.discourse.group/}{\textit{Vaultwarden}}.
		\end{Itemize}
		\vspace{-0.3cm}
		\textit{Σχετικές Δεξιότητες}: \textbf{Linux}, \textbf{Ansible}, \textbf{LEMP Stack}, \textbf{VPN}, \textbf{Bash}, \textbf{LDAP}, \textbf{PKI}, \textbf{Docker}, \textbf{Monitoring}
	\end{EntryDatedLogo}

	\section{Εκπαίδευση}
	\begin{EntryDatedLogo}{MEng ως Μηχανικός Πληροφορικής και Τηλεπικοινωνιών}{https://ece.uowm.gr}{Οκτώβριος 2014 -- Οκτώβριος 2023}{Τμήμα Ηλεκτρολόγων Μηχανικών και Μηχανικών Υπολογιστών, Πανεπιστήμιο Δυτικής Μακεδονίας}{assets/uowm-logo.png}{0.8}
		\begin{Itemize}
			\item Αποφοίτησα με Integrated Master's δίπλωμα (MΕng) ως Μηχανικός Πληροφορικής και Τηλεπικοινωνιών, με τελική βαθμολογία 7.39.
			\item Για τη \href{https://github.com/ChrisKar96/Thesis}{\textit{διπλωματική εργασία}} μου `Σχεδίαση και Υλοποίηση Πληροφοριακού Συστήματος Ανοιχτού Κώδικα για Διαχείριση Εκκίνησης Υπολογιστών μέσω Δικτύου', ανέπτυξα μια διαδικτυακή εφαρμογή βασισμένη σε PHP και MariaDB με την ονομασία \href{https://github.com/ChrisKar96/iBoot-Thesis}{\textit{iBoot}} χρησιμοποιώντας το \href{https://codeigniter.com/}{\textit{CodeIgniter 4}} MVC PHP framework, τη JavaScript βιβλιοθήκη \href{https://tabulator.info/}{\textit{Tabulator}} και το \href{https://swagger.io/specification/}{\textit{OpenAPI Specification}}.
		\end{Itemize}
	\end{EntryDatedLogo}

	\section{Πιστοποιήσεις}
    \begin{EntryDatedLogo}{Linux Foundation Certified Engineer (LFCE)}{https://www.credly.com/users/christos-karamolegkos/badges}{Μάρτιος 2022}{Linux Foundation}{assets/LFCE.png}{0.90}
		\begin{Itemize}
			\item Οι κάτοχοι αυτού του τίτλου αποδεικνύουν τις δεξιότητες και τις ικανότητες για την εκτέλεση των καθηκόντων ενός μηχανικού συστημάτων Linux, συμπεριλαμβανομένου του σχεδιασμού και της υλοποίησης της αρχιτεκτονικής του συστήματος. Οι κάτοχοι του τίτλου επέδειξαν επάρκεια στις βασικές εντολές, στη λειτουργία συστημάτων που λειτουργούν, στη διαχείριση χρηστών/ομάδων, στη δικτύωση, στη διαμόρφωση υπηρεσιών, στη διαχείριση αποθήκευσης, στο σχεδιασμό και την ανάπτυξη συστημάτων.
			\item Επίσης πέτυχα: Linux Foundation Certified Associate (LFCA) με ID Πιστοποίησης \textit{LF-lrlseaez8c}, Linux Foundation Certified System Administrator (LFCS) με ID Πιστοποίησης \textit{LF-wxclcuugus} και διάφορα εμβλήματα ολοκλήρωσης μαθημάτων όπως το `\href{https://www.credly.com/badges/19ff66ca-2e10-4e1b-90a9-1c1ac6132878}{\textit{LFS261: DevOps and SRE Fundamentals - Implementing Continuous Delivery}}' και το `\href{https://www.credly.com/badges/1fc7edfc-227e-4e93-ac46-297ab05c27db}{\textit{LFD201: Introduction to Open Source Development, GIT, and Linux}}'.
		\end{Itemize}
		\vspace{-0.3cm}
		Επικύρωση στο \href{https://training.linuxfoundation.org/certification/verify-linux-certifications}{\textit{training.linuxfoundation.org}}. ID Πιστοποίησης: \textit{LF-ok1fkcflc1}
	\end{EntryDatedLogo}

	\vspace{0.5cm}

    \begin{EntryDatedLogo}{GitLab Certified Associate}{https://about.gitlab.com/services/education/gitlab-certified-associate/}{Ιούλιος 2021}{GitLab}{assets/GitLab-Certified-Associate-2021-07-09.png}{0.90}
		\begin{Itemize}
			\item Άτομα που αποκτούν πιστοποίηση GitLab Certified Associate, μπορούν να εξηγήσουν τι είναι το GitLab και γιατί το χρησιμοποιούν οι ομάδες, να εκτελούν βασικές εντολές Git για branching, merging και remote work και να εφαρμόζουν θεμελιώδεις έννοιες και δεξιότητες χρησιμοποιώντας το GitLab στον κύκλο ζωής του DevOps.
		\end{Itemize}
		\vspace{-0.3cm}
		Επικύρωση στο \href{https://gitlab.badgr.com/public/assertions/Hw6j8Th9SyKNj8ehsQkqAw}{\textit{gitlab.badgr.com}}.
	\end{EntryDatedLogo}

	\vspace{0.5cm}

	\begin{EntryDatedLogo}{WSO2 Certified Identity Server
			Developer - V5}{https://wso2.com/training/certification/certified-identity-server-developer}{Ιούνιος 2020}{WSO2}{assets/wso2is-cert.pdf}{0.90}
		\begin{Itemize}
			\item Αυτή η πιστοποίηση έχει σχεδιαστεί για προγραμματιστές και αρχιτέκτονες εφαρμογών που έχουν θεμελιώδη γνώση των εννοιών IAM και πρακτική εμπειρία με τον WSO2 Ιdentity Server.
		\end{Itemize}
		\vspace{-0.3cm}
		Επικύρωση στο \href{https://certification.wso2.com}{\textit{certification.wso2.com}}. ID Πιστοποίησης: \textit{4TKN8V}
	\end{EntryDatedLogo}

	\section{Εθελοντική Εργασία}
	\begin{EntryDatedLogo}{Κέντρο Κοινωνικής Πρόνοιας Κεντρικής Μακεδονίας}{http://www.kkp-km.gr/}{2018 -- 2019}{Ανάπτυξη Λογισμικού}{assets/kkpkm.pdf}{0.75}
		\begin{Itemize}
			\item Ηγήθηκα του σχεδιασμού και της ανάπτυξης μιας διαδικτυακής εφαρμογής που παρακολουθεί τις θεραπευτικές διαδικασίες σε εκατοντάδες περιθαλπόμενους.
		\end{Itemize}
		\vspace{-0.3cm}
		Περισσότερες πληροφορείες στο site του  \href{https://diavgeia.gov.gr/decision/view/\%CE\%A8\%CE\%A6\%CE\%A1\%CE\%93\%CE\%9F\%CE\%9E\%CE\%A7\%CE\%A3-\%CE\%A0\%CE\%93\%CE\%A6}{\textit{diavgeia}}.
	\end{EntryDatedLogo}

\end{document}
