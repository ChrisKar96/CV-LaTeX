%!TeX spellcheck = el_GR-en_US
%!TEX TS-program = xelatex

\documentclass{mycv}

\RequirePackage[greek]{datetime2} 		 % show greek date correctly using \today command
\renewcommand{\today}{\ifcase\month\or
	Ιανουάριος\or Φεβρουάριος\or Μάρτιος\or% 
	Απρίλιος\or Μάιος\or Ιούνιος\or Ιούλιος%
	\or Αύγουστος\or Σεπτέμβριος\or% 
	Οκτώβριος\or Νοέμβριος\or% 
	Δεκέμβριος\fi\ \number \year}			 % nominative instead of genitive 
\RequirePackage{amsmath}

\hypersetup{%
	pdftitle={Χρήστος Καραμολέγκος - Βιογραφικό Σημείωμα},
	pdfauthor={Χρήστος Καραμολέγκος},
	pdfsubject={Βιογραφικό Σημείωμα},
	pdfkeywords={Χρήστος Καραμολέγκος,Βιογραφικό Σημείωμα,CV},
	pdflang={el-GR}
}

\begin{document}
	\pagestyle{empty}
	\name{Χρήστος Καραμολέγκος}{DevOps Engineer}{Systems Administrator}
	\contact{(+30)694-967-4952}{me@ckaramolegkos.gr}{ChrisKaramo}{www.ckaramolegkos.gr}{linkedin.com/in/chriskaramo}{ChrisKar96}
	\centering
	{\bf Ημερομηνία Γέννησης}: 24 Μαΐου 1996 {\Large\textperiodcentered} {\bf Τοποθεσία}: Αθήνα, Ελλάδα

	\section{Εισαγωγικό Σημείωμα}
	\textnormal Είμαι Μηχανικός DevOps και Διαχειριστής Πληροφοριακών Συστημάτων. Απέκτησα δίπλωμα Integrated Master ως Μηχανικός Πληροφορικής και Τηλεπικοινωνιών από το \href{https://ece.uowm.gr/?lan=gr}{\textit{τμήμα Ηλεκτρολόγων Μηχανικών και Μηχανικών Υπολογιστών του Πανεπιστημίου Δυτικής Μακεδονίας}}. Από νεαρή ηλικία ανακάλυψα το πάθος μου για την επίλυση προβλημάτων και τον σχεδιασμό λύσεων. Είμαι εργατικός, πάντα πρόθυμος να μαθαίνω και να επεκτείνω τις δυνατότητές μου, αξιοποιώντας τα πιο σύγχρονα εργαλεία και πληροφορίες. Είμαι δραστήριος και μπορώ να αποδίδω αποτελεσματικά ακόμη και υπό πίεση.

	\section{Εμπειρία}
	\begin{EntryDatedLogo}{Ελληνικός Στρατός}{http://army.gr}{Μαρτιος 2023 -- Δεκέμβριος 2023}{Υποχρεωτική Στρατιωτική Θητεία}{-0.3cm}{assets/hellenicarmy.png}{0.75}
		\vspace{-0.2cm}
		\begin{Itemize}
			\item Ανάπτυξη και συντήρηση κρυπτασφαλισμένης πλατφόρμας επικοινωνίας.
			\item Εκπαίδευση και υποστήριξη των χρηστών στη χρήση της πλατφόρμας.
		\end{Itemize}
	\end{EntryDatedLogo}

	\vspace{0.75cm}

	\begin{EntryDatedLogo}{Χίλων Πληροφορική}{http://www.hilonsys.com}{Μάιος 2022 -- Μάρτιος 2023}{DevOps Engineer}{-0.3cm}{assets/hilonsys.png}{0.75}
		\vspace{-0.2cm}
		\begin{Itemize}
			\item Υπεύθυνος για την υποδομή του οργανισμού και τη χρήση εργαλείων και διαδικασιών που σέβονται τις καλύτερες DevOps πρακτικές.
		\end{Itemize}
	\end{EntryDatedLogo}

	\vspace{0.75cm}

	\begin{EntryDatedLogo}{ΕΕΛΛΑΚ - ΟΡΓΑΝΙΣΜΟΣ ΑΝΟΙΧΤΩΝ ΤΕΧΝΟΛΟΓΙΩΝ}{https://eellak.gr}{Ιούλιος 2019 -- Μάρτιος 2023}{Systems Administrator}{-0.3cm}{assets/eellak.png}{0.75}
		\vspace{-0.2cm}
		\begin{Itemize}
			\item Υπεύθυνος για τη συντήρηση, τη διαμόρφωση και την αξιόπιστη λειτουργία της υποδομής του οργανισμού.
			\item Τα καθήκοντα της θέσης συμπεριλαμβάνουν τη διαχείριση βάσεων δεδομένων, δικτύου, ασφάλειας, ιστού και διακομιστών.
			\item Η εγκατάσταση αποτελείται από 100+ VMs, σε physical on-premise ή cloud hypervisors, κυρίως με \href{https://www.debian.org}{\textit{Debian}} OS, διαχειριζόμενα με χρήση του \href{https://debops.org}{\textit{DebOps}} ansible project.
			\item Εγκατάσταση, παραμετροποίηση και διαχείριση self-hosted υπηρεσιών όπως \href{https://wordpress.com}{\textit{WordPress}}, \href{https://moodle.org}{\textit{Moodle}}, \href{https://bigbluebutton.org}{\textit{BigBlueButton}}, \href{https://about.gitlab.com/install/}{\textit{GitLab}}, \href{https://nextcloud.com}{\textit{Nextcloud}}, \href{https://matrix.org}{\textit{Matrix}} / \href{https://element.io}{\textit{Element}} / \href{https://jitsi.org}{\textit{Jitsi}}.
		\end{Itemize}
	\end{EntryDatedLogo}

	\section{Δεξιότητες}
	\begin{tabular}{m{4.5cm} m{12.5cm}}
		\textbf{Διάγνωση / Επιδιόρθωση}     & Διάγνωση και επίλυση προβλημάτων υλικού και λογισμικού. \\
		\textbf{Διαχείριση Συστημάτων}		& Apache2/nginx, PKI, SSH, Email, LDAP, OpenVPN, VMware/Virtualbox, Ansible, High Availability \\
		\textbf{DevOps}	                    & Git, Github, GitLab, TravisCI, Docker \\
		\textbf{Προγραμματισμός} 	 	  	& Bash Scripting, Python, \LaTeX \\
		\textbf{Προγρ. Διαδικτύου}	  		& HTML/CSS, Bootstrap, Javascript, JQuery, Ajax, PHP, SQL, Wordpress \\
		\textbf{Λειτουργικά Συστήματα}   	& Linux, Windows \\
		\textbf{Διάφορα}        		 	& Σουίτες Office, Αλγοριθμική Σχεδίαση, Οργάνωση Project, Άδεια Οδήγησης Κατηγορίας Β \\
		\textbf{Γλώσσες} 			   		& Ελληνικά (μητρική γλώσσα), Αγγλικά 
	\end{tabular}

	\section{Εκπαίδευση}
	\begin{EntryDatedLogo}{Πανεπιστήμιο Δυτικής Μακεδονίας}{https://ece.uowm.gr}{Οκτώβριος 2014 -- Νοέμβριος 2023}{Τμήμα Ηλεκτρολόγων Μηχανικών και Μηχανικών Υπολογιστών}{0cm}{assets/uowm-logo.png}{0.8}
		\vspace{-0.2cm}
		\begin{Itemize}
			\item Έλαβα το Integrated Master δίπλωμα μου ως Μηχανικός Πληροφορικής και Τηλεπικοινωνιών.
		\end{Itemize}
	\end{EntryDatedLogo}

	\section{Πιστοποιήσεις}
    \begin{EntryDatedLogo}{Linux Foundation Certified Engineer (LFCE)}{https://www.credly.com/users/christos-karamolegkos/badges}{Μάρτιος 2022}{Linux Foundation}{-0.3cm}{assets/LFCE.png}{0.90}
    	\vspace{-0.2cm}
		\begin{Itemize}
			\item Οι κάτοχοι αυτού του τίτλου αποδεικνύουν τις δεξιότητες και τις ικανότητες για την εκτέλεση των καθηκόντων ενός μηχανικού συστημάτων Linux, συμπεριλαμβανομένου του σχεδιασμού και της υλοποίησης της αρχιτεκτονικής του συστήματος. Οι κάτοχοι του τίτλου επέδειξαν επάρκεια στις βασικές εντολές, στη λειτουργία συστημάτων που λειτουργούν, στη διαχείριση χρηστών/ομάδων, στη δικτύωση, στη διαμόρφωση υπηρεσιών, στη διαχείριση αποθήκευσης, στο σχεδιασμό και την ανάπτυξη συστημάτων.
			\item Επικύρωση στο \href{https://training.linuxfoundation.org/certification/verify-linux-certifications}{\textit{training.linuxfoundation.org}}. ID Πιστοποίησης: \textit{LF-ok1fkcflc1}
			\item Επίσης πέτυχα: Linux Foundation Certified Associate (LFCA) με ID Πιστοποίησης \textit{LF-lrlseaez8c}, Linux Foundation Certified System Administrator (LFCS) με ID Πιστοποίησης \textit{LF-wxclcuugus} και διάφορα εμβλήματα ολοκλήρωσης μαθημάτων όπως το "\href{https://www.credly.com/badges/19ff66ca-2e10-4e1b-90a9-1c1ac6132878}{\textit{LFS261: DevOps and SRE Fundamentals - Implementing Continuous Delivery}}" και το "\href{https://www.credly.com/badges/1fc7edfc-227e-4e93-ac46-297ab05c27db}{\textit{LFD201: Introduction to Open Source Development, GIT, and Linux}}".
		\end{Itemize}
	\end{EntryDatedLogo}

	\vspace{0.5cm}

    \begin{EntryDatedLogo}{GitLab Certified Associate}{https://about.gitlab.com/services/education/gitlab-certified-associate/}{Ιούλιος 2021}{GitLab}{-0.3cm}{assets/GitLab-Certified-Associate-2021-07-09.png}{0.90}
    	\vspace{-0.2cm}
		\begin{Itemize}
			\item Άτομα που κερδίζουν πιστοποίηση GitLab Certified Associate, μπορούν να εξηγήσουν τι είναι το GitLab και γιατί το χρησιμοποιούν οι ομάδες, να εκτελούν βασικές εντολές Git για branching, merging και remote work και να εφαρμόζουν θεμελιώδεις έννοιες και δεξιότητες χρησιμοποιώντας το GitLab στον κύκλο ζωής του DevOps.
			\item Επικύρωση στο \href{https://gitlab.badgr.com/public/assertions/Hw6j8Th9SyKNj8ehsQkqAw}{\textit{gitlab.badgr.com}}.
		\end{Itemize}
	\end{EntryDatedLogo}

	\vspace{0.5cm}

	\begin{EntryDatedLogo}{WSO2 Certified Identity Server
			Developer - V5}{https://wso2.com/training/certification/certified-identity-server-developer}{Ιούνιος 2020}{WSO2}{-0.3cm}{assets/wso2is-cert.pdf}{0.90}
		\vspace{-0.2cm}
		\begin{Itemize}
			\item Αυτή η πιστοποίηση έχει σχεδιαστεί για προγραμματιστές και αρχιτέκτονες εφαρμογών που έχουν θεμελιώδη γνώση των εννοιών IAM και πρακτική εμπειρία με τον WSO2 Ιdentity Server. 
			\item Επικύρωση στο \href{https://certification.wso2.com}{\textit{certification.wso2.com}}. ID Πιστοποίησης: \textit{4TKN8V}
		\end{Itemize}
	\end{EntryDatedLogo}

	\section{Εθελοντική Εργασία}
	\begin{EntryDatedLogo}{Κέντρο Κοινωνικής Πρόνοιας Κεντρικής Μακεδονίας}{http://www.kkp-km.gr/}{2018 -- 2019}{Ανάπτυξη Λογισμικού}{-0.3cm}{assets/kkpkm.pdf}{0.75}
		\vspace{-0.2cm}
		\begin{Itemize}
			\item Βασικός developer και σχεδιαστής ενός web application που καταγράφει και διαχειρίζεται τις θεραπείες των περιθαλπόμενων ενός ιδρύματος.
			\item Περισσότερες πληροφορείες στο site του  \href{https://diavgeia.gov.gr/decision/view/\%CE\%A8\%CE\%A6\%CE\%A1\%CE\%93\%CE\%9F\%CE\%9E\%CE\%A7\%CE\%A3-\%CE\%A0\%CE\%93\%CE\%A6}{\textit{diavgeia}}.
		\end{Itemize}
	\end{EntryDatedLogo}

\end{document}
